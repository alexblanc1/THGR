\begin{flushleft}

\begin{tikzpicture}
\clip (-10,-7) rectangle (10,7);  % Agrandir le cadre
%%%%%%%%%%%%%%%%%%%%%%%%%%%%%%%%%%%%%%%%%%%%%%%%%%%%%%%%%%%%%%%%%%%%%%%%%%%%%%%%%%%
    % Dessiner l'arc de cercle (grande partie) en pointillé
    %\draw[line width = 3pt, dash pattern=on 0pt off 12pt, line cap=round] (zero) ++(130:\r+0.6) arc [start angle=130, end angle=-95, radius=\r+0.6];
%%%%%%%%%%%%%%%%%%%%%%%%%%%%%%%%%%%%%%%%%%%%%%%%%%%%%%%%%%%%%%%%%%%%%%%%%%%%%%%%%%%

\pgfmathsetmacro{\r}{4}

\begin{scope}[shift={(6,0)}] % Cercle de droite
	\draw (0,0)circle (\r);
    % Définition des nœuds :
    \node (avav) at ({315:\r+0.5}) {};
    \node (avant) at ({285:\r+0.5}) {};
    \node (spmdi) at ({255:\r + 0.5}) {$\sigma^{p-2}(i)$};
    \node (spmui) at ({225:\r + 0.6}) {$\sigma^{p-1}(i)$};
    \node (i) at ({195:\r + 0.3}) {i}; 
    \node (si) at ({165:\r + 0.4}) {$\sigma(i)$};
    \node (ssi) at ({135: \r +0.4}) {$\sigma^{2}(i)$};
    \node (apres) at ({105:\r+0.4}) {};
    \node (aprapr) at ({75:\r+0.4}) {};

    % Les flèches :
    \draw[line width = 0.75pt,->] (spmdi) to [bend left = 20] (spmui);
    \draw[line width = 0.75pt,->] (spmui) to [bend left = 20] (i);
    \draw[line width = 0.75pt,->] (i) to [bend left = 20] (si);
    \draw[line width = 0.75pt,->] (si) to [bend left = 20] (ssi);
    \draw[line width = 0.75pt,->] (ssi) to [bend left = 20] (apres);
	\draw[line width = 0.75pt,->] (avant) to [bend left = 20] (spmdi);    
    
    \draw [line width = 3pt, dash pattern = on 0pt off 12 pt, line cap=round](apres) to [bend left =15] (aprapr);
	\draw [line width = 3pt, dash pattern = on 0pt off 12 pt, line cap=round](avav) to [bend left =15] (avant);


\end{scope}

\begin{scope}[shift={(-6,0)}]%cerlce de gauche
	%Le cerlce et les pts
	\draw (0,0) circle (\r);
	\node (j) at ({15:\r+0.3}) {j};
	\node (sj) at ({-15:\r+0.4}) {$\sigma(j)$};
	\node (ssj) at ({-45:\r+0.4}) {$\sigma^2(j)$};
	\node (sqmuj) at ({45:\r+0.6}) {$\sigma^{q-1}(j)$};
	\node (sqmdj) at ({75:\r+0.6}) {$\sigma^{q-2}(j)$};
	\node (preced) at ({105:\r+0.6}) {};
	\node (prepre) at ({135:\r+0.6}) {};
	\node (suiv) at ({-75:\r+0.4}) {};
	\node (suisui) at ({-105:\r+0.6}) {};
	
	%Les flèches :
	\draw [line width = 0.75pt,->] (sqmdj) to [bend left = 20](sqmuj);
	\draw [line width = 0.75pt,->] (sqmuj) to [bend left = 20](j);
	\draw[line width = 0.75pt,->] (j) to [bend left = 20](sj) ;
	\draw [line width = 0.75pt,->] (ssj) to [bend left = 20] (suiv);
	\draw [line width = 0.75pt,->] (preced) to [bend left = 20] (sqmdj);	
	\draw [line width = 0.75pt,->] (sj) to [bend left = 20](ssj);
	
	\draw [line width = 3pt, dash pattern = on 0pt off 12 pt, line cap=round](prepre) to [bend left =15] (preced);
	\draw [line width = 3pt, dash pattern = on 0pt off 12 pt, line cap=round](suiv) to [bend left =15] (suisui);
\end{scope}

\begin{scope}[red]
%Les flèches du millieux
	\draw[line width = 0.75pt,->] (j) to[bend left = 15] node [midway,above]{$\sigma\circ\tau$} (si)  ;
	\draw[line width = 0.75pt,->] (i) to[bend left = 15] node [midway,below]{$\sigma\circ\tau$} (sj);
	
\begin{scope}[shift={(-6,0)}]%flèche de gauche rouge
	%Le cerlce et les pts
	\node (j1) at ({15:\r-0.3}) {};
	\node (sj1) at ({-15:\r-0.3}) {};
	\node (ssj1) at ({-45:\r-0.3}) {} ;
	\node (sqmuj1) at ({45:\r-0.3}){};
	\node (sqmdj1) at ({75:\r-0.3}){};
	\node (preced1) at ({105:\r-0.3}){};
	\node (prepre1) at ({135:\r-0.3}){};
	\node (suiv1) at ({-75:\r-0.3}){};
	\node (suisui1) at ({-105:\r-0.3}){};
	
	%Les flèches :
	\draw [line width = 0.75pt,->] (sqmdj1) to [bend right = 20](sqmuj1);
	\draw [line width = 0.75pt,->] (sqmuj1) to [bend right = 20](j1);
	\draw [line width = 0.75pt,->] (ssj1) to [bend right = 20] (suiv1);
	\draw [line width = 0.75pt,->] (preced1) to [bend right = 20] (sqmdj1);	
	\draw [line width = 0.75pt,->] (sj1) to [bend right = 20](ssj1);
	
	\draw [line width = 3pt, dash pattern = on 0pt off 12 pt, line cap=round](prepre1) to [bend right =15] (preced1);
	\draw [line width = 3pt, dash pattern = on 0pt off 12 pt, line cap=round](suiv1) to [bend right =15] (suisui1);
\end{scope}

\begin{scope}[shift={(6,0)}] % Cercle de droite

    % Définition des nœuds :
    \node (avav1) at ({315:\r-0.3}) {};
    \node (avant1) at ({285:\r-0.3}) {};
    \node (spmdi1) at ({255:\r -0.3}) {};
    \node (spmui1) at ({225:\r -0.3}) {};
    \node (i1) at ({195:\r -0.3}) {}; 
    \node (si1) at ({165:\r -0.3}) {};
    \node (ssi1) at ({135: \r -0.3}) {};
    \node (apres1) at ({105:\r-0.3}) {};
    \node (aprapr1) at ({75:\r-0.3}) {};

    % Les flèches :
    \draw[line width = 0.75pt,->] (spmdi1) to [bend right = 20] (spmui1);
    \draw[line width = 0.75pt,->] (spmui1) to [bend right = 20] (i1);
    \draw[line width = 0.75pt,->] (si1) to [bend right = 20] (ssi1);
    \draw[line width = 0.75pt,->] (ssi1) to [bend right = 20] (apres1);
	\draw[line width = 0.75pt,->] (avant1) to [bend right = 20] (spmdi1);    
    
    \draw [line width = 3pt, dash pattern = on 0pt off 12 pt, line cap=round](apres1) to [bend right =15] (aprapr1);
	\draw [line width = 3pt, dash pattern = on 0pt off 12 pt, line cap=round](avav1) to [bend right =15] (avant1);
\end{scope}



\end{scope}

\end{tikzpicture}

\end{flushleft}
